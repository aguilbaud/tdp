\documentclass{article}
\usepackage[francais]{babel}
\usepackage[utf8]{inputenc}  
\usepackage{listings}
\usepackage{graphicx}
\usepackage{color}
\usepackage{float}
\usepackage{algorithm}
\usepackage[noend]{algpseudocode}

% Style for c code
\definecolor{mygreen}{rgb}{0,0.6,0}
\definecolor{gray}{rgb}{0.5,0.5,0.5}
\definecolor{mymauve}{rgb}{0.58,0,0.82}
\definecolor{backcolour}{rgb}{0.95,0.95,0.92}

\lstdefinestyle{cstyle}{ 
  language=C,
  backgroundcolor=\color{backcolour},   
  basicstyle=\footnotesize,        % the size of the fonts that are used for the code
  captionpos=b,                    % sets the caption-position to bottom
  commentstyle=\color{mygreen},    % comment style
  deletekeywords={...},            % if you want to delete keywords from the given language
  escapeinside={\%*}{*)},          % if you want to add LaTeX within your code
  keepspaces=true,                
  keywordstyle=\color{blue},       % keyword style
  otherkeywords={*,...},           % if you want to add more keywords to the set
  numbers=left,                   
  numbersep=5pt,                   % how far the line-numbers are from the code
  numberstyle=\tiny\color{gray}, % the style that is used for the line-numbers
  rulecolor=\color{black},        
  showspaces=false,               
  showstringspaces=false,          % underline spaces within strings only
  showtabs=false,                  % show tabs within strings adding particular underscores
  stepnumber=1,                    % the step between two line-numbers. If it's 1, each line will be numbered
  stringstyle=\color{mymauve},     % string literal style
  tabsize=2,	                   % sets default tabsize to 2 spaces
  title=\lstname                   % show the filename of files included with \lstinputlisting; also try caption instead of title
}

\lstset{style=cstyle}

\title{Factorisation LU et descente-remontée}
\author{Benjamin \bsc{Angelaud} - Adrien \bsc{Guilbaud}}
\begin{document}
\maketitle

\section{Version MPI}
\subsection{Factorisation LU}
Dans un premier temps, nous créons un buffer local à chaque processus qui contient ses colonnes locales selon une distribution en serpentin. Cela permet de rassembler chaque bloc-colonne de façon contigüe en mémoire évitant ainsi les cach-miss provoqués par le parcours la matrice initiale.

\paragraph{}
Dans l'algo permettant la factorisation LU, (algo \ref{algo:lu}), nous parcourons tous les bloc-colonnes; quand un bloc appartient à notre processus, nous faisons un dgetrf entre le bloc courant et tous nos autres blocs situé à droite. Nous envoyons ensuite notre bloc courant à tous les processus possédant des blocs à droite pour qu'ils mettent à jour leurs blocs. 

\begin{algorithm}
  \caption{Facto LU MPI}\label{algo:lu}
  \begin{algorithmic}[1]
    \State $ loc\_k $ = $ 0 $
    \For{ $k = 0$ to $k=nb\_blocs\_colonnes-1$ }  
    \If {$ is\_local\_bloc(k)$ }
    \State $ mycblas\_dgetrf(local\_bloc[loc\_k]) $
    \State $ broadcast(local\_bloc[loc\_k]) $
    \State $ loc\_k $ = $ loc\_k + 1 $
    \Else
    \State $ recevoir(distant\_bloc) $
    \State $ mycblas\_dtrsm(distant\_bloc, local\_bloc) $
    \State $ mycblas\_dgemm(distant\_bloc, local\_bloc) $
    \EndIf
    \EndFor

  \end{algorithmic}
\end{algorithm}


\subsection{Descente-remontée}

\section{Améliorations possible}
La factorisation LU en MPI ne marche que si l'ordre de la matrice est divisible par la taille d'un bloc. Il faudrait donc gérer le cas où il reste un morceau de matrice à attribuer à un processus.
\end{document}