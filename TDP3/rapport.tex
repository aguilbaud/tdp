\documentclass{article}
\usepackage[francais]{babel}
\usepackage[utf8]{inputenc}  
\usepackage{listings}
\usepackage{graphicx}
\usepackage{color}
\usepackage{float}
\usepackage{algorithm}
\usepackage[noend]{algpseudocode}

% Style for c code
\definecolor{mygreen}{rgb}{0,0.6,0}
\definecolor{gray}{rgb}{0.5,0.5,0.5}
\definecolor{mymauve}{rgb}{0.58,0,0.82}
\definecolor{backcolour}{rgb}{0.95,0.95,0.92}

\lstdefinestyle{cstyle}{ 
  language=C,
  backgroundcolor=\color{backcolour},   
  basicstyle=\footnotesize,        % the size of the fonts that are used for the code
  captionpos=b,                    % sets the caption-position to bottom
  commentstyle=\color{mygreen},    % comment style
  deletekeywords={...},            % if you want to delete keywords from the given language
  escapeinside={\%*}{*)},          % if you want to add LaTeX within your code
  keepspaces=true,                
  keywordstyle=\color{blue},       % keyword style
  otherkeywords={*,...},           % if you want to add more keywords to the set
  numbers=left,                   
  numbersep=5pt,                   % how far the line-numbers are from the code
  numberstyle=\tiny\color{gray}, % the style that is used for the line-numbers
  rulecolor=\color{black},        
  showspaces=false,               
  showstringspaces=false,          % underline spaces within strings only
  showtabs=false,                  % show tabs within strings adding particular underscores
  stepnumber=1,                    % the step between two line-numbers. If it's 1, each line will be numbered
  stringstyle=\color{mymauve},     % string literal style
  tabsize=2,	                   % sets default tabsize to 2 spaces
  title=\lstname                   % show the filename of files included with \lstinputlisting; also try caption instead of title
}

\lstset{style=cstyle}

\title{Factorisation LU et descente-remontée}
\author{Benjamin \bsc{Angelaud} - Adrien \bsc{Guilbaud}}
\begin{document}
\maketitle

\section{Version Séquentielle}
\subsection{Implémentation de dgetf2\_nopiv()}
\paragraph{}La première partie de ce TDP consistait à implémenter une factorisation séquentielle. La première fonction à réaliser a donc été dgetf2\_nopiv(), permettant d'obtenir une factorisation L.U à partir d'une matrice A.  Pour cela, nous réalisons un appel à 2 autres fonctions qui sont dscal() et dger(), que nous avons aussi implémenté. Dans la fonction dger() nous avons fait le choix de parcourir sur la boucle intérieur les lignes et les colonnes sur la boucle extérieur pour améliorer nos performances. En effet, comme nous sommes dans le cas où les matrices sont stockées en "column major", il est préférable de stocker une colonne dans le cache et la réutiliser pour le calcul de chaque ligne, plutôt que de devoir recharger toutes les colonnes pour chaque ligne, sachant que les données d'une ligne ne sont pas contiguës en mémoire. dgetf2\_nopiv() nous donne donc une factorisation LU séquentielle.

\subsection{Implémentation de dtrsm()}En gardant en tête notre objectif final, qui est de résoudre une équation linéaire, la fonction suivante à implémenter était donc dtrsm(), permettant, à partir de la factorisation LU, de résoudre Ax=B. dtrsm() va nous permettre de réaliser le produit d'une matrice triangulaire (supérieure ou inférieure) avec une matrice. Pour se faire, nous avons donc fait le choix d'implémenter la fonction dtrsv() qui permet de réaliser le produit d'une matrice triangulaire (supérieure ou inférieure) avec un vecteur, pour ensuite l'appeler sur chaque vecteur composant notre matrice.

\subsection{Implémentation de dgetrf()}Pour pouvoir avoir une factorisation LU par bloc il nous a fallu implémenter une nouvelle fonction réutilisant celles précédemment implémentées. Ici le but est simple, on fixe une taille de bloc, on effectue un dgetf2\_nopiv() sur les SIZE\_BLOCK première colonnes de la matrice, ensuite on effectue un dtrsm() pour mettre à jour nos SIZE\_BLOCK premières lignes, et pour finir on met à jour le reste de notre matrice avec un dgemm(). Comme nous l'avonc vu précédemment, nous avons implémenté dgetf2\_nopiv() et dtrsm() lors de ce tdp. Pour l'appel à dgemm, nous avons réutilisé notre fonction dgemm\_scalaire mise en place lors du TDP1. Nous sommes conscient que le fait d'utiliser dgemm scalaire nous fait perdre
énormément de performances comparé à une version en blocs, mais lors du TDP1 notre version blocs n'étant pas fonctionnelle nous avons donc réutilisé notre version scalaire.

\paragraph{}Nous avons rencontré beaucoup de problèmes sur cette partie de l'implémentation car nos tests de validité n'était pas correct. Néanmoins, nous avons réussi à implémenter une version de dgetrf() totalement fonctionnelle et sur n'importe quelle taille de bloc choisie, nous permettant ainsi de comparer les performances selon la taille des blocs choisie.

\subsection{Implémentation de dgesv()}Cette fonction est la fonction "finale" de notre factorisation séquentielle. Elle permet de faire la factorisation LU de notre matrice et de résoudre notre système linéaire par descente/remontée. Ici, nous faisons donc appelle à dgetrf() pour obtenir une factorisation LU de notre matrice (préférée à dgetf2\_nopiv() pour des raisons de performances), ensuite nous utilisons dtrsm() deux fois, la première pour obtenir le résultat de y dans Ly=b, puis pour obtenir x dans Ux=y. Une fois ceci terminé nous avons donc résolue notre système linéaire.

\section{Version MPI}
\subsection{Factorisation LU}
Dans un premier temps, nous créons un buffer local à chaque processus qui contient ses colonnes locales selon une distribution en serpentin. Cela permet de rassembler chaque bloc-colonne de façon contigüe en mémoire évitant ainsi les cach-miss provoqués par le parcours la matrice initiale.

\paragraph{}
Dans l'algo permettant la factorisation LU, (algo \ref{algo:lu}), nous parcourons tous les bloc-colonnes; quand un bloc appartient à notre processus, nous faisons un dgetrf entre le bloc courant et tous nos autres blocs situé à droite. Nous envoyons ensuite notre bloc courant à tous les processus possédant des blocs à droite pour qu'ils mettent à jour leurs blocs. 

\begin{algorithm}
  \caption{Facto LU MPI}\label{algo:lu}
  \begin{algorithmic}[1]
    \State $ loc\_k $ = $ 0 $
    \For{ $k = 0$ to $k=nb\_blocs\_colonnes-1$ }  
    \If {$ is\_local\_bloc(k)$ }
    \State $ mycblas\_dgetrf(local\_bloc[loc\_k]) $
    \State $ broadcast(local\_bloc[loc\_k]) $
    \State $ loc\_k $ = $ loc\_k + 1 $
    \Else
    \State $ recevoir(distant\_bloc) $
    \State $ mycblas\_dtrsm(distant\_bloc, local\_bloc) $
    \State $ mycblas\_dgemm(distant\_bloc, local\_bloc) $
    \EndIf
    \EndFor

  \end{algorithmic}
\end{algorithm}


\subsection{Descente-remontée}

\section{Améliorations possible}
La factorisation LU en MPI ne marche que si l'ordre de la matrice est divisible par la taille d'un bloc. Il faudrait donc gérer le cas où il reste un morceau de matrice à attribuer à un processus.
\end{document}