\documentclass{article}

\begin{document}
gregoire.pichon@inria.fr
Deadline 5 novembre, minuit

n particules plan 2D
mémoire limitée par processus

*une masse mi
*une position vec(pt(Mi)) 
*une vitesse vec(vt(Mi))

\paragraph{}
$ F_t(M_i, M_j) = F_t(M_j, M_i) * \overrightarrow{u}_{ij} $
\paragraph{}
$ F_t(M_i, M_j) = G \frac{m_i m_j}{(M_i M_j)^2} * \overrightarrow{u}_{ij} $
\paragraph{}
uij unitaire dirigé de Mi à Mj
\paragraph{}
$ F_t(M_i) = \sum_{i=1}^{n}  F_t(M_i, M_j) i \not= j $
\paragraph{}
$ \overrightarrow{a_t(M_i)} = \frac{F_t(M_i)}{m_i} $
\paragraph{}
Discrétitation avec un pas de temps dt: 
$ \overrightarrow{pt+dt}(M_i) = \overrightarrow{pt}(M_i) + \overrightarrow{vt}(M_i)dt + \frac{\overrightarrow{at}(M_i)}{2} * dt^2 $
\paragraph{}
vt+dt(Mi)  =vt(Mi) + at(Mi)dt


communication Persitante mieux ?
\end{document}
